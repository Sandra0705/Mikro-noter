%\documentclass[danish,12pt,a4paper]{report}
\documentclass[danish,12pt,a4paper]{report}

\usepackage[utf8]{inputenc}					
\usepackage[danish]{babel}				% Dokumentets sprog
\usepackage[T1]{fontenc}		
\usepackage{amsmath,amsthm,amssymb,amsfonts,bm}
\usepackage{ragged2e,anyfontsize}
\usepackage{etex}
\usepackage[titletoc]{appendix}
\usepackage{thmtools}
\usepackage[makeroom]{cancel}
\usepackage{placeins}
\usepackage{multirow}
\usepackage{mathtools}
\usepackage[danish]{varioref}
\usepackage[numbers]{natbib}
\usepackage{dirtytalk}
\usepackage{blindtext}
\RequirePackage{etex}
\usepackage{graphicx}
\usepackage{flafter}
\usepackage{float}
\usepackage{lastpage}
\usepackage{fancyhdr}
\usepackage{etoolbox}
\usepackage{mdframed}
\usepackage{enumitem}
\usepackage[a4paper]{geometry}
\usepackage{a4wide}
\usepackage{parskip}
\usepackage{booktabs}
\usepackage{rotating}
\usepackage{colortbl}
\usepackage{textcomp}
\usepackage{tabularx}
\usepackage{url}
\usepackage{hyperref}
\usepackage{tasks}
\usepackage{commath}
\usepackage{calc}
\usepackage{indentfirst}
\usepackage{caption}
\usepackage[official]{eurosym}
\usepackage{listings}
\renewcommand{\lstlistingname}{Kildekode}
\RequirePackage{silence}
\WarningFilter{remreset}{The remreset package}


%Fikser underfull og overfull
\hbadness=10001
\vbadness=10001

%Bibliografi og litteratur
\bibliographystyle{Formalia/vancouver}
\setcitestyle{square}
\usepackage[nottoc,numbib]{tocbibind}

%Gør indholdsfortegnelse og bibliografi dansk
\addto\captionsdanish{
	\renewcommand\contentsname{Indholdsfortegnelse}	
	\renewcommand{\bibname}{Bibliografi}
}

%%%% ORDDELING %%%%
\hyphenation{In-te-res-se e-le-ment}

%Sidehoved
\setlength{\headheight}{15pt}
\pagestyle{fancy}
\fancyhf{}
\renewcommand{\chaptermark}[1]{ \markboth{\thechapter.\ #1}{}}
\fancyheadoffset{0pt}
\lhead{\nouppercase \leftmark}
\rhead{Aalborg Universitet}
\renewcommand{\chaptermark}[1]
        {\markboth{#1}{}}
\renewcommand{\sectionmark}[1]
        {\markright{\thesection\ #1}}
\lfoot[\fancyplain{}{\bfseries\thepage}]
    {\fancyplain{}{}}
\rfoot[\fancyplain{}{}]%
    {\fancyplain{}{\bfseries\thepage}}
\patchcmd{\chapter}{plain}{fancy}{}{}

%Kapiteludseende
\usepackage{xcolor}
\usepackage{titlesec, blindtext, color}
\definecolor{gray75}{gray}{0.75}
\newcommand{\hsp}{\hspace{20pt}}
\titleformat{\chapter}[hang]{\huge\bfseries}{\thechapter\hsp\textcolor{gray75}{|}\hsp}{0pt}{\huge\bfseries}
\titlespacing*{\chapter}{0pt}{5pt}{25pt}

% Define a simple command to use at the start of a table row to make it have a shaded background
\newcommand{\gray}{\rowcolor[gray]{.9}}

\usepackage{textcomp}
\usepackage{url}
\usepackage{hyperref}

%TikZ
\usepackage{tikz}
\usetikzlibrary{arrows, petri, topaths,graphs,graphs.standard,arrows.meta}
\tikzstyle{arrow} = [thick,->,>=stealth]
\usepackage{tkz-berge}
\usepackage[position=top]{subfig}
\usepackage{verbatim}
\usepackage{pgfplots}
\pgfplotsset{compat=1.15}

%---- pseudocode 
\usepackage{algorithm}
\usepackage[noend]{algpseudocode}

\usepackage{framed}
\definecolor{myGray}{HTML}{F9F9F9}
\renewenvironment{leftbar}[4][\hsize]
{\def\FrameCommand
    {{\color{#2}\vrule width #4pt}
        \hspace{-8pt}
        \fboxsep=\FrameSep\colorbox{#3}}
    \MakeFramed{\hsize#1\advance\hsize-\width\FrameRestore}}
{\endMakeFramed}

\algnewcommand\algorithmicforeach{\textbf{for each}}
\algdef{S}[FOR]{ForEach}[1]{\algorithmicforeach\ #1\ \algorithmicdo}

%Sætninger, definitioner, mm. general stil
\declaretheoremstyle[
    % spaceabove=14pt, 
    % spacebelow=6pt, 
    headfont=\normalfont\bfseries, 
    bodyfont = \normalfont,
    postheadspace=2mm, 
    headpunct={.}]{mystyle}
    
%Sætning    
\declaretheorem[name={Sætning}, style=mystyle,numberwithin=section]{thm}
\newenvironment{thmx}[1]
    {\begin{leftbar}{black}{myGray}{3}\begin{thm}#1}{\end{thm}\end{leftbar}}
%Definition
\declaretheorem[name={Definition}, style=mystyle,sibling=thm]{defni}
\newenvironment{defn}[1]
    {\begin{leftbar}{black}{myGray}{3}\begin{defni}#1}{\end{defni}\end{leftbar}}
%Eksempel
\declaretheorem[name={Eksempel}, style=mystyle,sibling=thm]{exmp}
\newenvironment{eks}[1]
    {\begin{leftbar}{gray}{white}{3}\begin{exmp}#1}{\end{exmp}\end{leftbar}}
%Lemma
\declaretheorem[name={Lemma}, style=mystyle,sibling=thm]{lema}
\newenvironment{lem}[1]
    {\begin{leftbar}{black}{myGray}{3}\begin{lema}#1}{\end{lema}\end{leftbar}}
%Proposition
\declaretheorem[name={Proposition}, style=mystyle,sibling=thm]{prop}
\newenvironment{pro}[1]
    {\begin{leftbar}{black}{myGray}{3}\begin{prop}#1}{\end{prop}\end{leftbar}}
%Korollar
\declaretheorem[name={Kildekode}, style=mystyle,sibling=thm]{koro}
\newenvironment{kor}[1]
    {\begin{koro}#1}{\end{koro}}

%Bevis
\declaretheoremstyle[
    spaceabove=14pt, 
    spacebelow=6pt, 
    headfont=\normalfont\bfseries, 
    bodyfont = \normalfont,
    postheadspace=1mm, 
    qed=$\blacksquare$, 
    headpunct={.}]{bevisstyle}
\declaretheorem[name={Bevis}, style=bevisstyle,numbered=no]{bev}

%Inline graphics
\newlength\myheight
\newlength\mydepth
\settototalheight\myheight{Xygp}
\settodepth\mydepth{Xygp}
\setlength\fboxsep{0pt}
\newcommand*\inlinegraphics[1]{%
  \settototalheight\myheight{Xygp}%
  \settodepth\mydepth{Xygp}%
  \raisebox{-\mydepth}{\includegraphics[height=\myheight]{#1}}}


%Kommandoer, som gør jeres liv nemmere, når I skriver. Pas på med at lave for mange kommandoer selv
%da det kan være træls for jer når I skal indsende MWE (minimal working examples) ind i fx stackexchange
\usepackage{bbm}
\newcommand{\R}{\mathbb{R}}
\newcommand{\1}{\mathbbm{1}}
\newcommand{\Z}{\mathbb{Z}}
\newcommand{\N}{\mathbb{N}}
\renewcommand{\d}{\mathrm{d}}
\newcommand{\eps}{\varepsilon}
\newcommand{\e}{\mathrm{e}}
\newcommand{\E}{\mathcal{E}}
\newcommand{\tr}{\mathrm{tr }}
\newcommand{\F}{\mathbb{F}}
\renewcommand{\L}{\mathcal{L}}
\newcommand{\m}{\cdot}

\def\partautorefname{Del}
\def\chapterautorefname{Kapitel}
\def\sectionautorefname{Afsnit}
\def\subsectionautorefname{Underafsnit}
\def\figureautorefname{Figur}
\def\tableautorefname{Tabel}
\def\algorithmautorefname{Algoritme}
\def\lstinputlistingautorefname{Kildekode}